\section{Cardinality}

\begin{theorem}[theorem 2.12 in baby rudin]
Let $\{ E_n \},n=1,2,\dots$ be a sequence of countable sets, and put
\[
S=\bigcup_{n=1}^{\infty} E_n
\]
Then $S$ is countable.\label{d81540}
\end{theorem}

\begin{figure}[H]
\centering
\includegraphics[width=\textwidth]{Cardinality-20250312.png}
% \caption{}
\label{}
\end{figure}

\cref{d81540} can be written as
\[
\underbrace{ \aleph_0+\dots+\aleph_0 }_{ \aleph_0 }=\aleph_0\times\aleph_0=\aleph_0
\]
\begin{theorem}[theorem 2.13 in baby rudin]
Let $A$ be a countable set, and let $B_n$ be the set of all $n$ -tuples $(a_1,\dots,a_n)$, where $a_k\in A$ ($k=1,\dots,n$) and the elements $a_1,\dots,a_n$ need not be distinct. Then $B_n$ is countable.\label{c78d3f}
\end{theorem}

\cref{c78d3f} is proved by induction from \cref{d81540}, which means
\[
\underbrace{ \aleph_0\times\aleph_0\times\dots \times\aleph_0 }_{ n }=\aleph_0
\]
But
\[
\underbrace{ \aleph_0\times\aleph_0\times\dots \times\aleph_0 }_{ \aleph_0 }=\aleph_0^{\aleph_0}\sim \mathbb{R}
\]
In particular, we have

\begin{theorem}[theorem 2.14 in baby rudin]
Let $A$ be the set of all sequences whose elements are the digits $0$ and $1$. This set $A$ is uncountable.
The elements of $A$ are sequences like $1,0,0,1,0,1,1,1,\dots$
\end{theorem}
The proof is very classic to show that $A$ is uncountable.

Let $E$ be a countable subset of $A$, and let $E$ consist of the sequences $s_1,s_2,s_3,\dots$. We construct a sequence $s$ as follows. If the $n$ th digit in $s_n$ is $1$, we let the $n$ th digit of $s$ be $0$, and vice versa. \textbf{Then the sequence $s$ differs from every member of $E$ in at least one place}; hence $s\not\in E$. But clearly $s\in A$, so that $E$ is a proper subset of $A$.

We have shown that every countable subset of $A$ is a proper subset of $A$. It follows that $A$ is uncountable (for otherwise $A$ would be a proper subset of $A$, which is absurd).

\subsubsection{Example}

It should be noted that the mapping
\[
\{ 0,1 \}^{\infty}\to[0,1]\qquad (n_1,n_2,\dots)\mapsto \sum_{k=1}^{\infty}\frac{n_k}{2^{k}} 
\]
is surjective but not injective. We denote the sets by
\[
A=\{ (\dots,1,\underbrace{ 0 }_{ i },1,\dots),(\dots,0,\underbrace{ 1 }_{ i },0,\dots):i\in \mathbb{N} \}
\]
\[
B=\left\{  \frac{n}{2^{m}} :m\in \mathbb{N},n\in \mathbb{N},n\leq 2^{m}  \right\}
\]
It's easy to check that $A,B$ are countable. Then
\[
\{ 0,1 \}^{\infty}\setminus A\to[0,1]\setminus B\qquad (n_1,n_2,\dots)\mapsto \sum_{k=1}^{\infty}\frac{n_k}{2^{k}} 
\]
is injective, thus bijective. Therefore
\[
\{ 0,1 \}^{\infty}\sim A\cup(\{ 0,1 \}^{\infty}\setminus A)\sim \{ 0,1 \}^{\infty}\setminus A\sim [0,1]\setminus B\sim B\cup([0,1]\setminus B)\sim [0,1]\sim \mathbb{R}
\]