\section{刷题:Lebesgue测度-可测集-测度-Borel集}

参见《实变函数解题指南》·周民强.

\subsection{Lebesgue 测度}

\begin{definition}[Lebesgue outer measure]
Let $E \subset \mathbf{R}^n$. If $\left\{I_k\right\}$ is a countable collection of open rectangles in $\mathbf{R}^n$, and
\[
E \subset \bigcup_{k \geqslant 1} I_k,
\]then $\left\{I_k\right\}$ is called an $L$-covering of $E$ (obviously there are many such coverings, and each $L$-covering $\left\{I_k\right\}$ determines a non-negative generalized real value $\sum_{k \geqslant 1}\left|I_k\right|$ (can be $+\infty$, $\left|I_k\right|$ represents the volume of $I_k$), we call
\[
m^*(E)=\inf \left\{\sum_{k \geqslant 1}\left|I_k\right|:\left\{I_k\right\} \text { is an } L-\text { covering of } E\right\}
\]the \textbf{Lebesgue outer measure} of the point set $E$.
\end{definition}
\begin{exercise}
(5)设 $E \subset[a, b], m^*(E)>0,0<c<m^*(E)$ ,则存在 $E$ 的子集 $A$ ,使得 $m^*(A)=c$ .
\end{exercise}
\begin{proof}
Let $f(x)=m^*([a, x) \cap E), a \leq x \leq b$. Then $f(a)=0, f(b)=$ $m^*(E)$. Consider $x$ and $x+\Delta x$. Without loss of generality, assume that $a \leq x<x+\Delta x \leq b$. Then, by
\[
[a, x+\Delta x) \cap E=([a, x) \cap E) \cup([x, x+\Delta x) \cap E),
\]
we have $f(x+\Delta x) \leq f(x)+\Delta x$, that is,
\[
f(x+\Delta x)-f(x) \leq \Delta x .
\]
Similarly, for $\Delta x<0$, we can also prove similar inequalities. In summary, we have
\[
|f(x+\Delta x)-f(x)| \leq|\Delta x|, \quad a \leq x \leq b .
\]
This indicates that $f \in C([a, b])$. According to the intermediate value theorem for continuous functions, for $f(a)<c<f(b)$, there must exist $\xi \in(a, b)$ such that $f(\xi)=c$. Thus, taking $A=[a, \xi) \cap E$ yields the desired result.
\end{proof}

\begin{definition}[可数覆盖空间]
设 $X$ 是一个拓扑空间. 如果存在一个可数集 $A \subset X$, 使得 $X$ 的任何开覆盖都有一个子覆盖, 其元素个数不超过 $A$ 的元素个数, 那么 $X$ 被称为\textbf{可数覆盖空间}.
\end{definition}
一个拓扑空间 $X$ 被称为\textbf{可数覆盖空间}, 如果存在一个可数子集 $A \subset X$, 使得 $X$ 的任何开覆盖都有一个子覆盖, 其元素个数不超过 $A$ 的元素个数. 换句话说, 存在一个可数集合 $A$, 使得对于 $X$ 的任何开覆盖, 都可以从中选择一个可数子覆盖.
由于 $\mathbb{R}^n$ 具有可数基, 因此 $\mathbb{R}^n$ 是第二可数空间. 根据定义, 第二可数空间是 Lindelöf 空间, 这意味着 $\mathbb{R}^n$ 的每个开覆盖都有一个可数子覆盖. 因此, $\mathbb{R}^n$ 是一个可数覆盖空间.

\begin{exercise}
(5)设 $E \subset \mathbf{R}^n$ .若对任意的 $x \in E$ ,存在开球 $B\left(x, \delta_x\right)$ ,使得 $m^*\left(E \cap B\left(x, \delta_x\right)\right)=0$ ,则 $m^*(E)=0$ .
\end{exercise}
\begin{proof}
(5) By assumption, there exists a countable cover of $E$ by balls $\left\{B_k \triangleq B\left(x_k, \delta_{x_k}\right)\right\}$ such that $E \subset \bigcup_{k=1}^{\infty} B_k$, and $m^*\left(E \cap B_k\right)=0$. It follows that
\[
E=\bigcup_{k=1}^{\infty}\left(E \cap B_k\right), \quad m^*(E) \leqslant \sum_{k=1}^{\infty} m^*\left(E \cap B_k\right)=0 .
\]
\end{proof}

\begin{proposition}
设 $A, B \subset \mathbf{R}^n$ ,且 $m^*(A), m^*(B)<\infty$ ,则
\[
\left|m^*(A)-m^*(B)\right| \leqslant m^*(A \triangle B) ;
\]
\end{proposition}
\begin{definition}[Limit superior and Limit inferior of sets]
Let $A_1, A_2, \dots$ be a sequence of sets. The \textbf{limit superior} is defined by
\[
\limsup _{n \rightarrow \infty} A_n \equiv\left\{x: x \in A_n \text { for infinitely many } n\right\}=\bigcap_{n=1}^{\infty} \bigcup_{i=n}^{\infty} A_i
\]The \textbf{limit inferior} is defined by
\[
\liminf _{n \rightarrow \infty} A_n \equiv\left\{x: x \in A_n \text { for all but finitely many } n\right\}=\bigcup_{n=1}^{\infty} \bigcap_{i=n}^{\infty} A_i
\]
\end{definition}
\begin{exercise}
(2)设 $E_k \subset \mathbf{R}^n(k \in \mathbf{N})$ .若 $\sum_{k=1}^{\infty} m^*\left(E_k\right)<+\infty$ ,则 $m\left(\overline{\lim _{k \rightarrow \infty}} E_k\right)=0$ .
(3)设定义在 $\mathbf{R}^1$ 上的函数列 $\left\{f_n(x)\right\}$ 满足 $\left(\lambda_n>0, n \in \mathbf{N}\right)$
\[
\sum_{n=1}^{\infty} m^*\left(E_n\right)<+\infty \quad\left(E_n=\left\{x \in \mathbf{R}^1:\left|f_n(x)\right| / \lambda_n>1\right\}\right),
\]则存在 $Z \subset \mathbf{R}^1$ 且 $m(Z)=0$ ,使得 $\varlimsup_{n \rightarrow \infty}\left|f_n(x)\right| / \lambda_n \leqslant 1\left(x \in \mathbf{R}^1 \backslash Z\right)$ .
\end{exercise}
\begin{proof}
(2)注意 $\varlimsup_{k \rightarrow \infty} E_k=\bigcap_{m=1}^{\infty} \bigcup_{k=m}^{\infty} E_k$ ,且依题设知,对任给 $\varepsilon>0$ ,存在 $N$ ,使得 $\sum_{k=N}^{\infty} m^*\left(E_k\right)<\varepsilon$ .从而对任意 $j \in \mathbf{N}$ ,有
\[
m^*\left(\varlimsup _{k \rightarrow \infty} E_k\right) \leqslant m^*\left(\bigcup_{k=j}^{\infty} E_k\right) \leqslant \sum_{k=j}^{\infty} m^*\left(E_k\right) .
\]
由此知 $m^*\left(\varlimsup _{k \rightarrow \infty} E_k\right) \leqslant \sum_{k=N}^{\infty} m^*\left(E_k\right)<\varepsilon$ .证毕.

(3)令 $Z=\bigcap_{m=1}^{\infty} \bigcup_{n=m}^{\infty} E_n$ ,则由题设知 $m(Z)=0$ .因此当 $x \in \mathbf{R}^1 \backslash Z$ 时,必存在 $n_0$ ,使得 $x \bar{\in} E_n\left(n \geqslant n_0\right)$ .从而有 $\varlimsup_{n \rightarrow \infty}\left|f_n(x)\right| / \lambda_n \leqslant 1\left(x \in \mathbf{R}^1 \backslash Z\right)$ .
\end{proof}

\subsection{可测集与测度}

\begin{definition}[Lebesgue 可测集,Carathéodory 条件]
设 $E \subset \mathbf{R}^n$ .若对任意的点集 $T \subset \mathbf{R}^n$ ,有
\[
m^*(T)=m^*(T \cap E)+m^*\left(T \cap E^{c}\right),
\]则称 $E$ 为 \textbf{Lebesgue 可测集}(或 $m^*$ -可测集),简称为\textbf{可测集},其中 $T$ 称为\textbf{试验集} (这一定义可测集的等式也称为 \textbf{Carathéodory 条件});可测集的全体称为\textbf{可测集类},简记为 $\mathcal{M}$ .
\end{definition}
\begin{proposition}
设 $\left\{E_k\right\}$ 是可测集列,则 $m\left(\varliminf _{k \rightarrow \infty} E_k\right) \leqslant \varliminf_{k \rightarrow \infty} m\left(E_k\right)$ .
\end{proposition}
\begin{exercise}
设 $E\subset \mathbb{R}^{n}$, $E\in \mathcal{M}$ 的充要条件是:对任给 $\epsilon>0$, 存在可测集 $A, B\in \mathbb{R}^{n}$: $A\subset E\subset B$, 使得 $m(B\setminus A)<\epsilon$.
\end{exercise}
\begin{exercise}
(3)设 $\left\{E_n\right\}$ 是 $[0,1]$ 中的可测集列,且满足 $\overline{\lim }_{n \rightarrow \infty} m\left(E_n\right)=1$ ,试证明对任意的 $\alpha: 0<\alpha<1$ ,必存在 $\left\{E_{n_k}\right\}$ ,使得 $m\left(\bigcap_{k=1}^{\infty} E_{n_k}\right\}>\alpha$ .
\end{exercise}
\begin{proof}
(3)由题设知,对任意的 $k \in \mathbf{N}$ ,存在 $\left\{n_k\right\}$ ,使得
\[
m\left(E_{n_k}\right)>1-(1-\alpha) / 2^k \quad(k \in \mathbf{N}) .
\]
由此知 $1-m\left(E_{n_k}\right)<(1-\alpha) / 2^k(k \in \mathbf{N})$ .从而得到
\[
\begin{aligned}
& {[0,1] \backslash \bigcap_{k=1}^{\infty} E_{n_k}=\bigcup_{k=1}^{\infty}\left([0,1] \backslash E_{n_k}\right),} \\
& m\left([0,1] \backslash \bigcap_{k=1}^{\infty} E_{n_k}\right) \leqslant \sum_{k=1}^{\infty} m\left([0,1] \backslash E_{n_k}\right) \\
& \quad=\sum_{k=1}^{\infty}\left(1-m\left(E_{n_k}\right)\right) \leqslant \sum_{k=1}^{\infty}(1-\alpha) / 2^k=1-\alpha,
\end{aligned}
\]
故有 $m\left(\bigcap_{k=1}^{\infty} E_{n_k}\right)>\alpha$ .
\end{proof}

\begin{exercise}
(5)设有 $\mathbf{R}^1$ 中可测集列 $\left\{E_k\right\}$ ,且当 $k \geqslant k_0$ 时,$E_k \subset[a, b]$ .若存在 $\lim _{k \rightarrow \infty} E_k=E$ ,试证明:$m(E)=\lim _{k \rightarrow \infty} m\left(E_k\right)$ .
\end{exercise}
\begin{proof}
\[
m(\lim_{ k \to \infty } E_k)\geq \varlimsup_{ k \to \infty } m(E_k)\geq \varliminf_{ k \to \infty } m(E_k)\geq m(\lim_{ k \to \infty } E_k)
\]
\end{proof}

\begin{exercise}
构造零测的第二纲集.
\end{exercise}
\begin{proof}
(2)令 $[0,1] \cap \mathbf{Q}=\left\{r_1, r_2, \cdots, r_n, \cdots\right\}$ ,以及
\[
I_{n, k}=\left(r_n-\frac{1}{2^{n+k}}, r_n+\frac{1}{2^{n+k}}\right) \quad(n, k \in \mathrm{~N}),
\]
则点集 $(-\infty,+\infty) \backslash \bigcup_{n, k=1}^{\infty} I_{n, k}$ 在 $\mathbf{R}^1$ 中无处稠密. 我们有
\[
m\left(\bigcap_{k=1}^{\infty} \bigcup_{n=1}^{\infty} I_{n, k}\right)=0,
\]
$\bigcap_{k=1}^{\infty} \bigcup_{n=1}^{\infty} I_{n, k}$ 是第二纲集.
\end{proof}

\begin{exercise}
(3)在 $[0,1]$ 中作点集
$E=\left\{x \in[0,1]:\right.$ 在十进位小数表示式 $x=0 . a_1 a_2 \cdots$ 中的所有 $a_i$ 都不出现 10 个数字中的某一个 $\}$ ,
试证明 $E$ 是不可数集,且 $m(E)=0$ .
\end{exercise}
\begin{proof}
Let $E_k$ be a set defined as follows:
\[
E_k = \{x \in [0,1]: \text{ in the decimal expansion } x = 0.a_1 a_2 \dots, \text{ the digit } k \text{ does not appear for any } a_i \}
\]
where $k \in \{0, 1, 2, \dots, 9\}$.

The set $E_k$ is constructed by iteratively removing intervals from $[0,1]$. In the first step, the interval $[0,1]$ is divided into 10 equal subintervals, and the $(k+1)$-th subinterval $[\frac{k}{10}, \frac{k+1}{10})$ is removed. This ensures that the remaining points have a decimal expansion where the first digit is not equal to $k$.

In the second step, each of the remaining 9 subintervals is divided into 10 equal subintervals, and the $(k+1)$-th subinterval is removed from each. This process continues indefinitely.

The total length of the removed intervals can be calculated as the sum of a geometric series:
\[
\frac{1}{10} + \frac{9}{10^2} + \frac{9^2}{10^3} + \dots = \frac{1}{10} \sum_{n=0}^{\infty} \left(\frac{9}{10}\right)^n = \frac{1}{10} \cdot \frac{1}{1 - \frac{9}{10}} = \frac{1}{10} \cdot \frac{1}{\frac{1}{10}} = 1
\]
Since the total length of the intervals removed is 1, the measure of the remaining set $E_k$ is:
\[
m(E_k) = 1 - 1 = 0
\]
\end{proof}

\begin{exercise}
(5)将 $[0,1]$ 中的点用十进位小数展开,令
\[
E=\{x \in[0,1]: x \text { 的任一位小数是 } 2 \text { 或 } 7\} \text {, }
\]试问:(i)$E$ 是闭集?(ii)$E$ 是开集?(iii)$E$ 是可数集?(iv) $\bar{E}=[0,1]$ ? (v)$E$ 是可测集?$m(E)=$ ?
\end{exercise}
\begin{proof}
(5) (i) Suppose $\left\{x_k\right\} \subset E: x_k \rightarrow x(k \rightarrow \infty)$, and let $x=\sum_{n=1}^{\infty} b_n / 10^n$. If $\left|x_k-x\right|<1 / 10^p$, then $b_p=2$ or 7. Hence $x \in E$, i.e., $E$ is a closed set.

(ii) Noticing (i) and $E \neq[0,1]$, we know that $E$ is not an open set.

(iii) $\#E =2^{\aleph_0}=\aleph_1$.

(iv) Since $E$ is a closed set and $E \neq[0,1]$, we know that $E$ is not dense in $[0,1]$.

(v) $E$ is a measurable set (see $\S 2.3$ ). $m(E)=1-0.8 \times \sum_{m=0}^{\infty}(2 / 10)^m=0$.

\end{proof}

\begin{exercise}
(1)设 $E \subset \mathbf{R}^1$ ,且存在 $q: 0<q<1$ ,使得对任一区间 $(a, b)$ ,都有开区间列 $\left\{I_n\right\}$ :
\[
E \cap(a, b) \subset \bigcup_{n=1}^{\infty} I_n, \quad \sum_{n=1}^{\infty} m\left(I_k\right)<(b-a) q .
\]则 $m(E)=0$ .
\end{exercise}
\begin{proof}
证明(1)因为 $m^*(E)=m^*\left(E \cap \bigcup_{n=1}^{\infty} I_n\right) \leqslant \sum_{n=1}^{\infty} m^*\left(E \cap I_n\right)$ ,所以只需指出对任意的 $(a, b)$ ,有 $m^*(E \cap(a, b))=0$ .由题设知,存在 $I_n=$ $\left(a_n, b_n\right)(n \in \mathbf{N}), \bigcup_{n=1}^{\infty} I_n \supset E \cap(a, b)$ ,使得 $\sum_{n=1}^{\infty}\left(b_n-a_n\right) \leqslant q(b-a)$ .再对每个 $\left(a_n, b_n\right)$ 作覆盖,其总长度小于 $q\left(b_n-a_n\right)$ .依此程序继续作下去,可得 (对任意 $k \in \mathbf{N}$ )
\[
m^*(E \cap(a, b)) \leqslant q \sum_{n=1}^{\infty}\left(b_n-a_n\right) \leqslant q^2(b-a) \cdots \leqslant q^k(b-a),
\]
由此易知 $m^*(E \cap(a, b))=0$ .

\end{proof}

\begin{exercise}[yau-21-analysis-problem-3]
\begin{enumerate}
		\item For $f \in L^1(\mathbb{R}^n)$, $g \in L^\infty(\mathbb{R}^n)$, show that their convolution $f * g$ is a well-defined continuous function.
		\item Let $E \subset \mathbb{R}^n$ be a Lebesgue measurable set with Lebesgue measure $m(E) > 0$. Prove that
\[
E - E := \{ x - y \mid x \in E, y \in E \}
\]contains an open neighborhood of $0 \in \mathbb{R}^n$.
	\end{enumerate}
\end{exercise}
\begin{proof}
(a) This is standard: In fact, we have $\|f * g\|_{L^\infty} \leq \|f\|_{L^1} \|g\|_{L^\infty}$. Therefore, by the continuity argument, it suffices to prove the theorem for $f \in C_0^\infty(\mathbf{R}^n)$. In this case, we have
\[
|f * g(x_0 + x) - f * g(x_0)| = \left| \int_{\mathbf{R}^n} (f(x_0 + x - y) - f(x_0 - y)) g(y) \, dy \right|
\]
\[
\leq \|g\|_{L^\infty} \int_{\mathbf{R}^n} |f(x_0 + x - y) - f(x_0 - y)| \, dy
\]
Now let $x \to 0$, the integrand converges to 0 uniformly. This yields (a).

(b) It suffices to consider the case where $m(E) < \infty$. We take $f = \mathbf{1}_E$, $g = \mathbf{1}_{-E}$, thus $h(x) = f * g$ is a continuous function. In particular, $h(0) = m(E) > 0$. Therefore, there exists an open set $U$ such that $0 \in U$ and $h|_U > \delta > 0$ for some $\delta > 0$. For $x \in U$, by definition,
\[
h(z) = \int_{\mathbf{R}^n} \mathbf{1}_E(x - y) \mathbf{1}_{-E}(y) \, dy > 0.
\]
Therefore, there must be some $y \in -E$, such that $x - y = x + (-y) \in E$. This implies $x \in E - (-y) \subset E - E$. Hence $U \subset E - E$.
\end{proof}
